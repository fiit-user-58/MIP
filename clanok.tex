\documentclass{article}
\usepackage[slovak]{babel}
\usepackage[IL2]{fontenc}
% \usepackage[utf8]{inputenc}
\usepackage{graphicx}
\usepackage{doi}
% \usepackage{url}
% \usepackage{hyperref}
% \usepackage{enumitem}

\title{Uncovering the Ethics Behind Recommendation Systems}
\author{Serhii\;Zadorozhnyi}
% \date{}
\begin{document}

\maketitle
\textbf{}
(EN)
\begin{abstract}
Nowadays, recommendation systems (RS) play significant part in our lives, they have many various applications such as healthcare, education, and news, they play a critical role across various digital platforms, from e-commerce to social media. They decide for us what to watch, what to buy, there to go on vacation, that may be interesting for us and more. Having a good system of recommendations is a huge benefit for any store or social media, as it can significantly enhance user engagement, drive sales, and improve customer satisfaction by providing personalized content that matches individual preferences and needs. In this paper we will talk about RS, starting with their beginning and ending up nowadays, we will discuss their impact on user experience, ethical side of RS like unfair recommendations caused by algorithmic bias, manipulation and how recommendations subtly influence our behavior and decision-making, and the transparency of algorithms, which is essential in RS to promote user trust and clarity about origin of their recommendations, also we will discuss growing role of AI and machine learning in RS, highlighting both their potential benefits and risks for society. In the end, we will think about future developments in RS, which may include enhanced fairness algorithms and more user-centric personalization approaches and transparency.
\end{abstract}
(SK)
\begin{abstract}
Odporúčacie systémy (OS) dnes zohrávajú významnú úlohu v našom živote, majú mnoho rôznych aplikácií, ako napríklad v zdravotníctve, vzdelávaní a spravodajstve, a majú rozhodujúcu úlohu na rôznych digitálnych platformách, od elektronického obchodu až po sociálne médiá. Rozhodujú za nás, čo sledovať, čo kúpiť, kam ísť na dovolenku, čo nás môže zaujímať a podobne. Mať dobrý systém odporúčaní je obrovským prínosom pre každý obchod alebo sociálne médiá, pretože môže výrazne zvýšiť zapojenie používateľov, podporiť predaj a zlepšiť spokojnosť zákazníkov poskytovaním personalizovaného obsahu, ktorý zodpovedá individuálnym preferenciám a potrebám. V tomto článku budeme hovoriť o OS, od ich začiatkov az po súčasnosť, budeme diskutovať o ich vplyve na používateľskú skúsenosť, o etickej stránke OS, ako sú nespravodlivé odporúčania spôsobené algoritmickym posunom, o manipuláci a to, ako odporúčania nenápadne ovplyvňujú naše správanie a rozhodovanie, a o transparentnosti algoritmov, ktorá je v RS nevyhnutná na podporu dôvery používateľov a jasnosti pôvodu vygenerovanych odporúčaní, tiež budeme diskutovať o rastúcej úlohe umelej inteligencie v OS, pričom poukážeme na ich potenciálne prínosy aj riziká pre spoločnosť. Na záver sa zamyslíme nad budúcim vývojom v OS, ktorý môže zahŕňať vylepšené algoritmy spravodlivosti a prístupy personalizácie a transparentnosti viac zamerané na používateľa.    
\end{abstract}
% \section{USE LATER}
% About users who often provide personal data without fully understanding how it is used.

% \section{Induction}
% Recommendation systems (RS) 
% \section{Background}
% \section{Ethical Challenges}
\section{Zdroje}
zdroj1\cite{9904386}\\zdroj2\cite{7956539}
\bibliography{lit}
\bibliographystyle{IEEEtran}
\end{document}
