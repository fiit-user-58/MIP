\documentclass{article}
\usepackage[]{babel}
\usepackage[IL2]{fontenc}
% \usepackage[utf8]{inputenc}
\usepackage{graphicx}
\usepackage{doi}
% \usepackage{url}
\usepackage{hyperref}
% \usepackage{enumitem}

\title{Uncovering the Ethics Behind Recommendation Systems}
\author{Serhii\;Zadorozhnyi}
% \date{}
\begin{document}

\maketitle
\textbf{}

\begin{abstract}
Nowadays, recommendation systems (RS) play significant part in our lives, they have many various applications such as healthcare, education, and news, they play a critical role across various digital platforms, from e-commerce to social media. They decide for us what to watch, what to buy, there to go on vacation, that may be interesting for us and more. Having a good system of recommendations is a huge benefit for any store or social media, as it can significantly enhance user engagement, drive sales, and improve customer satisfaction by providing personalized content that matches individual preferences and needs. In this paper we will talk about RS, starting with their beginning and ending up nowadays, we will discuss their impact on user experience, ethical side of RS like unfair recommendations caused by algorithmic bias, manipulation and how recommendations subtly influence our behavior and decision-making, and the transparency of algorithms, which is essential in RS to promote user trust and clarity about origin of their recommendations, also we will discuss growing role of AI and machine learning in RS, highlighting both their potential benefits and risks for society. In the end, we will think about future developments in RS, which may include enhanced fairness algorithms and more user-centric personalization approaches and transparency.
\end{abstract}

% \section{USE LATER}
% About users who often provide personal data without fully understanding how it is used.

\section{Introduction}
Recommendation systems (RS) have now become an essential guide for decision-making on a broad spectrum of movie and book choices to restaurants and products in today's technology-merged life. Traditionally, recommendations were a result of personal insight and human interaction: one person would suggest an option to another, allowing for a nuanced exchange shaped by personal knowledge and context. However, with the inclusion of RS using AI, algorithms have been playing the lead role in compiling a personalized set of recommendations for the users.

This article summarizes the evolution of RS, starting from Elaine Rich's pioneer work on the Grundy system in 1979 \cite{RICH1979329} to use stereotypes for suggesting books to users. From that day forward, these systems underwent huge technological development and influenced users with their customized content. The RS become more pervasive, and with the development, critical ethical issues arise on issues such as transparency, bias, and user privacy. Furthermore, machine learning introduces potential benefits and risks, raising an overt need to approach ethical and transparent methods in system design.

In this article, we review the past, present, and future of recommendation systems; we focus on the need for interdisciplinary approaches in order to assess ethical and societal implications of algorithmic recommendations. 
\section{Evolution and Functionality of Recommendation Systems}
\section{Impact of Recommendation Systems on User Experience}
\section{Ethical Challenges}
\section{Transparency and Explainability in Recommendation Systems}
\section{AI and Machine Learning in Recommendation Systems: Benefits and Risks}
\section{Future of Ethical Recommendation Systems ????}
\section{Conclusion}
% zdroj1\cite{9904386}\\zdroj2\cite{7956539}
\bibliography{lit}
\bibliographystyle{IEEEtran}
\end{document}
